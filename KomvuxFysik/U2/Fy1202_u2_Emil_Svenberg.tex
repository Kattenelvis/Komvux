\documentclass[a4paper,12pt]{article}
%For images
\usepackage{graphicx}
 
\addtolength{\oddsidemargin}{-.875in}
\addtolength{\evensidemargin}{-.875in}
\addtolength{\textwidth}{1.75in}
 
\addtolength{\topmargin}{-.875in}
\addtolength{\textheight}{1.75in}
 
\begin{document}
\begin{enumerate}
    \item Hastigheten vid x ledet blir följande om vi väljer ett 
    koordinat system där kanten av bordet är där x=0 och vi antar att 
    det ej finns acceleration på x-ledet.

    $$x=v_xt\Rightarrow v_x=\frac{x}{t}$$
    
    Då ser man att tiden t måste härledas.
    Y positionen följer också ekvationen för linjär rörelse, men 
    $v_{y}=0$ då bordet är platt och då saknar hastighet i
    y håller och accelerationen är gravitationskraften.

    $$y=-\frac{gt^2}{2}+h$$

    Vi sätter att $y=0$ för att ta reda på tiden
    när den når marken och så löser vi ut för t
    $$t=\sqrt{\frac{2h}{g}}$$

    Sedan sätter vi in det i den orginella ekvationen
    $$v_x=x\sqrt{\frac{g}{2h}}$$

    Jag har ett litet bord som är 0.3m högt och kulan landade 0.5 meter bort. 
    Då färdades den med hastigheten 
    $$0.5\sqrt{\frac{9.8}{2\cdot 0.3}}\approx 2 m/s$$

    Som då är frågans svar.

    \item Normalkraften är summan av centripetalaccelerationen och gravitationskraften.
          $$A_c-mg=N$$
          När N är mindre än noll så kommer bilen att stanna kvar på marken. Annars
          kommer bilen accelerera uppåt i luften.
          $$N=0 \Rightarrow mg=A_c$$
          $$mg=mv^2/r$$
          $$gr=v^2$$
          $$\sqrt{gr}=v$$
          Och med värderna g=9.8 och krökradien r=80 meter så får vi
          $$\sqrt{80*9.8}=28$$
          Alltså, så länge $v\leq 28$ så stannar bilen på marken. Så den högsta
          hastigheten den klarar av är då 28 m/s eller runt 100km/h.

    \item Ekvationen för det totala arbetet i en fjäder är följande (Fundamentals Of Physics, Jearl Walker, s. 422):
    $$W=\frac{1}{2}kx^2$$
    
    Denna ekvation innehåller fjäderkonstanten k som lär räknas ut först.
    Antalet Newtons som krävdes att ändra x med 0.35 
    ges i uppgiften som 105, så ekvationen $F=kx$ 
    blir $105=k\cdot 0.35\Rightarrow k=\frac{105}{0.35}=300$N/m.

    Då blir x i följande ekvation 0.5m vilket är en halvmeter. 
    $$W=\frac{1}{2}300\cdot 0.5^2=37,5$$

    Och svaret på frågan blir då 37.5 Joules.

    \item 
    \begin{enumerate}
        \item För att framställa ekvationen av typ $asin(\omega t)$ som visar fjärdens 
        elongation krävs det att $\omega$, 
        den rotationella hastigheten, löses ut. Detta kräver två fakta:
        $$\omega=\sqrt{\frac{k}{m}}\rightarrow\frac{1}{\omega}=\sqrt{\frac{m}{k}}$$
        Och
        $$T=2\pi\sqrt{\frac{m}{k}}=2\pi\frac{1}{\omega}$$
        Som då löser ut sig till
        $$\omega=2\pi\frac{1}{T}=2\pi f$$
        
        Detta ger oss då ekvationen för viktens position i fjädern som
        en funktion av tid
        $$a\sin(2\pi f\cdot t)$$
 
        Newtons andra lag säger att $F=ma=m(x)''$
        Så då deriveras ekvationen
        $$(x)'=2\pi f\cdot a\cos(2\pi f\cdot t)$$
        $$(x)''=-4\pi^2 f^2\cdot a\sin(2\pi f\cdot t)$$

        Avståndet från jämnvikten är 0.10 meter. Detta kan även beskrivas som
        $$a\sin(2\pi f\cdot t)=0.1$$
        Löser ut för t
        $$t=\frac{\sin^{-1}(0.1a^{-1})}{2\pi f}\approx\sin^{-1}(0.67)/(2\cdot 3.14 \cdot 1/1.5)\approx 0,17s$$
    
    
        Kraften vid den punkten blir då
        $$
        F=m(x)''\approx-0.350\cdot4\pi^2 \frac{1}{1.5^2}\cdot 0.15\sin(2\pi \frac{1}{1.5}\cdot 0.17)
        \approx -0.92\sin(0.71) \approx -0.61
        $$

        Kraften blir då alltså -0.61 Newtons.

        \item Vi vet att F=kx. 
        $$T=2\pi\sqrt{\frac{m}{k}}\Rightarrow k=4\pi^2\frac{m}{T^2}\approx 39.5\frac{0.350}{1.5^2}\approx 6,14$$
        Vi vet från uppgiften att $x=0.1$
        Så då blir kraften $0.1\cdot 6.14 = 0.614$ Newtons.

    \end{enumerate}

    \item 
    När två fjädrar seriekopplas så kommer längden som varje fjäder drar ner att halveras.
    Så då blir det x/2 istället för x. 
    om $mg/(x/2)=k$ och $\omega = \sqrt{\frac{k}{m}}$ så följer det att 
    $$\omega = \sqrt{\frac{mg/(x/2)}{m}}=\sqrt{\frac{2g}{x}}$$

    Så då blir svängningstiden och svaret på frågan
    $$T=2\pi\frac{1}{\omega}=2\pi\sqrt{\frac{x}{2g}}\approx 6.28\sqrt{\frac{0.12}{2*9.8}}\approx 0.49s$$

\end{enumerate}
\end{document}