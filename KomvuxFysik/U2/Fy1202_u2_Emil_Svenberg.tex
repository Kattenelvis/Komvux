\documentclass[a4paper,12pt]{article}
%For images
\usepackage{graphicx}
 
\addtolength{\oddsidemargin}{-.875in}
\addtolength{\evensidemargin}{-.875in}
\addtolength{\textwidth}{1.75in}
 
\addtolength{\topmargin}{-.875in}
\addtolength{\textheight}{1.75in}
 
\begin{document}
\begin{enumerate}
    \item hi

    \item Normalkraften är summan av centripetalaccelerationen och gravitationskraften.
          $$A_c+mg=N$$
          När N är mindre än noll så kommer bilen att stanna kvar på marken. Annars
          kommer bilen accelerera uppåt i luften.
          $$mg=A_c\rightarrow N=0$$
          $$mg=mv^2/r$$
          $$gr=v^2$$
          $$\sqrt{gr}=v$$
          Och med värderna g=9.8 och krökradien r=80 meter så får vi
          $$\sqrt{80*9.8}=28$$
          Alltså, så länge $v=28$ så stannar bilen på marken. Så den högsta
          hastigheten den klarar av är då 28 m/s eller runt 100km/h, för då är N=0
          och då kan ingenting hålla kvar bilen på marken.

    \item Ekvationen för det totala arbetet i en fjäder är följande (Fundamentals Of Physics, ,2000, s. 422):
    $$W=\frac{1}{2}kx^2$$
    
    Denna ekvation innehåller fjäderkonstanten k som lär räknas ut först.
    Antalet Newtons som krävdes är en bra ledtråd, då ekvationen $F=kx$ 
    blir $105=k\cdot 0.35$. 0.35 m = 35 centimeter och 105 är kraften i newton som 
    gavs i uppgiften. Då får man att $k=\frac{105}{0.35}=300$N/m.

    Då blir x i följande ekvation 0.5m vilket är en halvmeter. 
    $$W=\frac{1}{2}300\cdot 0.5^2=37,5$$

    Och svaret på frågan blir då 37.5 Joules.

    \item 
    \begin{enumerate}
        \item s
    \end{enumerate}
\end{enumerate}
\end{document}