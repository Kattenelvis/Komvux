\documentclass[a4paper,12pt]{article}
%For images
\usepackage{graphicx}
 
\addtolength{\oddsidemargin}{-.875in}
\addtolength{\evensidemargin}{-.875in}
\addtolength{\textwidth}{1.75in}
 
\addtolength{\topmargin}{-.875in}
\addtolength{\textheight}{1.75in}
 
\begin{document}
\begin{enumerate}
    \item hi

    \item Normalkraften är summan av centripetalaccelerationen och gravitationskraften.
          $$A_c+mg=N$$
          När N är mindre än noll så kommer bilen att stanna kvar på marken. Annars
          kommer bilen accelerera uppåt i luften.
          $$mg=A_c\rightarrow N=0$$
          $$mg=mv^2/r$$
          $$gr=v^2$$
          $$\sqrt{gr}=v$$
          Och med värderna g=9.8 och krökradien r=80 meter så får vi
          $$\sqrt{80*9.8}=28$$
          Alltså, så länge $v=28$ så stannar bilen på marken. Så den högsta
          hastigheten den klarar av är då 28 m/s eller runt 100km/h, för då är N=0
          och då kan ingenting hålla kvar bilen på marken.

    \item Ekvationen för det totala arbetet i en fjäder är följande (Fundamentals Of Physics, ,2000, s. 422):
    $$W=\frac{1}{2}kx^2$$
    
    Denna ekvation innehåller fjäderkonstanten k som lär räknas ut först.
    Antalet Newtons som krävdes är en bra ledtråd, då ekvationen $F=kx$ 
    blir $105=k\cdot 0.35$. 0.35 m = 35 centimeter och 105 är kraften i newton som 
    gavs i uppgiften. Då får man att $k=\frac{105}{0.35}=300$N/m.

    Då blir x i följande ekvation 0.5m vilket är en halvmeter. 
    $$W=\frac{1}{2}300\cdot 0.5^2=37,5$$

    Och svaret på frågan blir då 37.5 Joules.

    \item 
    \begin{enumerate}
        \item För att framställa ekvationen av typ $asin(\omega t)$ som visar fjärdens 
        elongation krävs det att $\omega$, 
        den rotationella hastigheten, löses ut. Detta kräver två fakta:
        $$\omega=\sqrt{\frac{k}{m}}\rightarrow\frac{1}{\omega}=\sqrt{\frac{m}{k}}$$
        Och
        $$T=2\pi\sqrt{\frac{m}{k}}=2\pi\frac{1}{\omega}$$
        Som då löser ut sig till
        $$\omega=2\pi\frac{1}{T}=2\pi f$$
        
        Detta ger oss då ekvationen för viktens position i fjädern som
        en funktion av tid
        $$a\sin(2\pi f\cdot t)$$
 
        Newtons andra lag säger att $F=ma=m(x)''$
        Så då deriveras ekvationen
        $$(x)'=2\pi f\cdot a\cos(2\pi f\cdot t)$$
        $$(x)''=-4\pi^2 f^2\cdot a\sin(2\pi f\cdot t)$$

        Avståndet från jämnvikten är 0.10 meter. Detta kan även beskrivas som
        $$a\sin(2\pi f\cdot t)=0.1$$
        Löser ut för t
        $$t=\frac{\sin^{-1}(0.1a^{-1})}{2\pi f}\approx\sin^{-1}(0.67)/(2\cdot 3.14 \cdot 1/1.5)\approx 0,17s$$
    
    
        Kraften vid den punkten blir då
        $$
        F=m(x)''\approx-0.350\cdot4\pi^2 \frac{1}{1.5^2}\cdot 0.15\sin(2\pi \frac{1}{1.5}\cdot 0.17)
        \approx -0.92\sin(0.71) \approx -0.61
        $$

        Kraften blir då alltså -0.61 Newtons.

        \item Vi vet att F=kx. 
        $$T=2\pi\sqrt{\frac{m}{k}}\Rightarrow k=4\pi^2\frac{m}{T^2}\approx 39.5\frac{0.350}{1.5^2}\approx 6,14$$
        Vi vet från uppgiften att $x=0.1$
        Så då blir kraften $0.1\cdot 6.14 = 0.614$ Newtons.

    \end{enumerate}

    \item om dom är likadana så antar jag att detta innebär att $k_1=k_2=k$
    och $a_1=a_2=-g$ och $x_1=x_2=x$. $-mg=kx$.  $mg/x=k$

\end{enumerate}
\end{document}