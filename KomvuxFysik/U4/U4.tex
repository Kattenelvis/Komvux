\documentclass[a4paper,12pt]{article}
%For images
\usepackage{graphicx}
 
\addtolength{\oddsidemargin}{-.875in}
\addtolength{\evensidemargin}{-.875in}
\addtolength{\textwidth}{1.75in}
 
\addtolength{\topmargin}{-.875in}
\addtolength{\textheight}{1.75in}
 
\begin{document}
\begin{enumerate}
    \item Enligt Faraday's lag är volten $e=Bvl$ med v som hastigheten
    l=1.2 som längden på kofångaren och B=0.050 som gjordens magnetfält,
    fast man multiplicerar med sinus av inklinationsvinkeln då
    man vill ha fram det som påverkar kofångaren. 
    Volten för ledlampor är ungefär 1.5 V. Då blir det

    $$v=\frac{e}{\sin(\theta)Bl}=\frac{1.5}{\sin(71)0.050\cdot 1.2}=26.4m/s=95.1km/h$$

    KOM IHÅG BILD HÄR XXXXXXXXXXXXXXXX
    \item Med hjälp av "bil-regeln" så får man fram att
    svaret blir $F=BIL=0.10\cdot 3.5\cdot 0.08=0.028$N.

    \item 

\end{enumerate}
\end{document}