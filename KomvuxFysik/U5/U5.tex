\documentclass[a4paper,12pt]{article}
%For images
\usepackage{graphicx}
 
\addtolength{\oddsidemargin}{-.875in}
\addtolength{\evensidemargin}{-.875in}
\addtolength{\textwidth}{1.75in}
 
\addtolength{\topmargin}{-.875in}
\addtolength{\textheight}{1.75in}
 
\begin{document}
\begin{enumerate}
      \item Om solen stråler som en svartkropp så kommer ekvationen för
            svartkoppsstrålning att hålla för värderna.
            Ekvationen för svartkropps strålning är $W=\sigma A T^4$.
            Solens temperatur är 5800K och arean är $4\pi r^2$.

            Om solen var en svartkropp skulle effekten bli
            $$\sigma 4\pi r^2T^4=5.671\cdot 10^{-8} \cdot 4 \cdot \pi \cdot (6.96\cdot 10^8)^2\cdot 5800^4=3.9066091632\cdot 10^{26}$$

            Vilket stämmer väldigt klart överens med det värdet som finns i uppgiften. Därför
            kan man slutsatsen att solen är ungefär en svartkoppsstrålning

      \item
            \begin{enumerate}
                  \item För denna uppgift kan Wiens Försjutningslag användas.
                        $$\lambda=0.00290/T$$
                        I Kelvin så är temperaturen 33+273.15=306.15.
                        $$\lambda=0.00290/306.15\approx 9.472\cdot 10^{-6}$$

                        Detta är rimligt då talet hamnar inom den infraröda delen
                        av det elektromagnetiska spektrumet, vilket är precis där
                        man förväntar sig att detta sker.

                  \item
            \end{enumerate}


      \item Med Niehls Bohr ekvation
            $$h\cdot f =\Delta E$$
            och från hastigheten av vågor som åker i ljusets hastighet
            $f=\frac{c}{\lambda}$ får man
            $$\Delta E=\frac{hc}{\lambda}=\frac{6.62607\cdot 10^{-34}\cdot 300\cdot 10^6}{576.960\cdot 10^{-9}}=3.44534\cdot 10^{-19}$$
            Joules av energi, vilket är
            $$\frac{3.44534\cdot 10^{-19}}{1.6\cdot 10^{-19}}=2.153 eV$$

      \item
            \begin{enumerate}
                  \item



                  \item



            \end{enumerate}
\end{enumerate}
\end{document}