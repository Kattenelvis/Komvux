\documentclass[a4paper,12pt]{article}
%For images
\usepackage{graphicx}
 
\addtolength{\oddsidemargin}{-.875in}
\addtolength{\evensidemargin}{-.875in}
\addtolength{\textwidth}{1.75in}
 
\addtolength{\topmargin}{-.875in}
\addtolength{\textheight}{1.75in}
 
\begin{document}
\begin{enumerate}
      \item Om solen stråler som en svartkropp så kommer ekvationen för
            svartkoppsstrålning att hålla för värderna.
            Ekvationen för svartkropps strålning är $W=\sigma A T^4$.
            Solens temperatur är 5800K och arean är $4\pi r^2$.

            Om solen var en svartkropp skulle effekten bli
            $$\sigma 4\pi r^2T^4=5.671\cdot 10^{-8} \cdot 4 \cdot \pi \cdot (6.96\cdot 10^8)^2\cdot 5800^4\approx 3.91\cdot 10^{26}$$

            Vilket stämmer väldigt klart överens med det värdet som finns i uppgiften. Därför
            kan man slutsatsen att solen är ungefär en svartkoppsstrålning

      \item
            \begin{enumerate}
                  \item För denna uppgift kan Wiens Försjutningslag användas.
                        $$\lambda=0.00290/T$$
                        I Kelvin så är temperaturen 33+273.15=306.15.
                        $$\lambda=0.00290/306.15\approx 9.472\cdot 10^{-6}$$

                        Detta är rimligt då talet hamnar inom den infraröda delen
                        av det elektromagnetiska spektrumet, vilket är precis där
                        man förväntar sig att detta sker.

                  \item
            \end{enumerate}


      \item Med Niehls Bohr ekvation
            $$h\cdot f =\Delta E$$
            och från hastigheten av vågor som åker i ljusets hastighet
            $f=\frac{c}{\lambda}$ får man
            $$\Delta E=\frac{hc}{\lambda}=\frac{6.62607\cdot 10^{-34}\cdot 300\cdot 10^6}{576.960\cdot 10^{-9}}=3.44534\cdot 10^{-19}$$
            Joules av energi, vilket är
            $$\frac{3.44534\cdot 10^{-19}}{1.6\cdot 10^{-19}}=2.153 eV$$

      \item
            \begin{enumerate}
                  \item Enligt lagen om energins bevarande så måste
                        energin i elektronen efter fotonen blev absorberad $E_f$ vara lika med
                        energin innan $E_i$ adderat med fotonens energi $E_p$.
                        $$E_f=E_i+E_p\Rightarrow E_p=E_f-E_i$$

                        Energin innan baseras på ekvationen $-\frac{13.6}{n^2}$ där
                        n=2 eftersom att det enda exciterade tillståndet som går i en
                        heliumatom är andra skalet. Energin efteråt står i uppgiften
                        som 5.6 eV. Då blir svaret

                        $$E_p=5.6-(-\frac{13.6}{2^2})=9eV$$

                  \item Denna uppgift svaras som ovanstående uppgift fast med n=1
                        $$E_p=5.6-(-\frac{13.6}{1^2})=-8eV$$
                        Men eftersom att negativ energi är omöjligt så går detta inte.
                        Det visar på att atomern ej joniserades.
            \end{enumerate}

      \item 
      \begin{enumerate}
            \item 
            $$h\cdot c=1.98645\cdot 10^{-25}$$
            Energin mellan A och grundtillståndet är
            $$E_{AG}=\frac{hc}{\lambda}=\frac{1.98645\cdot 10^{-25}}{2.4\cdot 10^{-7}}=8.3\cdot 10^{-19}J=5.2eV$$
            
            Energin mellan A till C är.
            $$E_{AC}=\frac{hc}{\lambda}=\frac{1.98645\cdot 10^{-25}}{4.1\cdot 10^{-7}}=4.8\cdot 10^{-19}J=3.0eV$$
            
            Kollar man på bilden i uppgiften bör pilarna 
            bli från C till grundtillståndet vara pilen från
            A till grundtillståndet minus A till C. 
            Då blir det 
            $$E_{AG}-E_{AC}=5.2-3=2.2eV=3.5\cdot 10^{-19}J$$
            
            För att få fram våglängden så ställs ekvationen om
            $$\lambda=\frac{hc}{E}=\frac{1.98645\cdot 10^{-25}}{3.5\cdot 10^{-19}}=563nm$$
            
            Vilket är svaret på uppgiften.

            \item 
            Den som tar det från B till grundtillståndet borde
            räknas ut med liknande metoder som ovan. Här bör man ta
            $E_{AG}-E_{AC}+E_{BC}$.
            
            $$E_{BC}=\frac{hc}{\lambda}=\frac{1.98645\cdot 10^{-25}}{12.4\cdot 10^{-7}}=1.6\cdot 10^{-19}J=0.9eV$$
      
            $$E_{AG}-E_{AC}+E_{BC}=2.2+0.9=3.2eV$$
      
      \end{enumerate}
      \item 
      \begin{enumerate}
            \item  
            Detta handlar om rödförsjutning och jag använder formulan
            $$z=\frac{\lambda-\lambda_0}{\lambda_0}=\frac{486.157\cdot 10^{-9}-486.133\cdot 10^{-9}}{486.133\cdot 10^{-9}}\approx 4.937\cdot 10^{-5}$$

            \item Om den uppmätta våglängden är längre så vet man att stjänran
            är påväg ifrån jorden. Eftersom att svaret blev positivt så vet vi att den är påväg från jorden.
      \end{enumerate}

\end{enumerate}
\end{document}