\documentclass[a4paper,12pt]{article}
%For images
\usepackage{graphicx}
 
\addtolength{\oddsidemargin}{-.875in}
\addtolength{\evensidemargin}{-.875in}
\addtolength{\textwidth}{1.75in}
 
\addtolength{\topmargin}{-.875in}
\addtolength{\textheight}{1.75in}
 
\begin{document}
\begin{enumerate}
    \item \begin{enumerate}
        \item Med produktregeln blir derivatan
        $$x'cos(x)+x(cos(x))'=cos(x)-xsin(x)$$

        \item Med produktregeln blir derivatan
        $$\frac{e^x}{x}-\frac{e^x}{x^2}$$
    \end{enumerate}

    \item \begin{enumerate}
        \item Derivatan är 

        \item $$y'(x)=-\frac{4}{x^2}+\frac{1}{x}$$
        Så 
        $$y'(2)=-\frac{4}{4}+\frac{1}{2}=-0.5$$

        Vilket kan verifieras med en bild

        \begin{center}
            \includegraphics[scale=0.5]{Figur 1.jpg}
        \end{center}
    \end{enumerate}

    \item
    Derivatan blir
    $$\frac{10}{x}-2x+8$$

    Som illustreras som röd medans den svarta är orginalekvationen.
    
    \begin{center}
        \includegraphics[scale=0.6]{Figur 2.jpg}
    \end{center}
    
    Extrempunkterna hamnar där derivatan är noll
    $$\frac{10}{x}-2x+8=0$$
    Skriver om den till en andragradsekvation
    $$10-2x^2=-8x\Rightarrow x^2-4x-5=0\Rightarrow (x-5)(x+1)=0$$
    Extrempunkterna sker då vid 5 och -1.
    Man kan se vid dom olika omskrivningarna av derivatan
    så behåller möter dom fortfarande varandra på x-axeln.
    Att manipulera polynomer som man sätter lika med noll är samma
    som att transformera kurvan i det tvådimensionella planet som behåller
    punkten.

    \begin{center}
        \includegraphics[scale=0.6]{Figur 3.jpg}
    \end{center}

    Men eftersom att den orginella ekvationen inehåller
    $10ln(x)$ så är den odefinierad för reella variabler.
    Då lär jag svara att det finns en extrempunkt vid x=5

    Tar man andraderivatan $-\frac{10}{x^2}-2$ vid x=5 får man att 
    andraderivatan är negativ, vilket innebär att det är en maximipunkt.   
    Detta kan man se i figur 2. 

    \item 
    Ekvationen för en tangent vid punkt $x_0$ är
    $f'(x_0)x+f(x_0)-f'(x_0)x_0$.
    
    Eftersom att $x_0=0$ så simplifieras den till
    $f'(0)x+f(0)$.

    $f(0)=4\sin(2\cdot 0)+\cos(0)=1$

    $f'(x)=8cos(2x)-sin(x)$
    
    $f'(0)=8cos(2\cdot 0)-sin(0)=8$

    Så då blir tangenten och svaret på frågan följande:
    $f'(0)x+f(0)=8x+1$

    I figuren nedan är tangenten blå, derivatan lila 
    och den orginella ekvationen är röd. Observera att 
    y och x axlarna inte har samma skala.

    \begin{center}    
        \includegraphics[scale=0.5]{Figur4.png}
    \end{center}

    \item Asymptoterna är ställen i funktionen där man delar på noll.
    Vid $$y=\frac{2}{x-3}+4$$ är det självklart att 3 är en assymptot.
    På y-axeln sker det också en asymptot. 4:an i ekvationen
    får en att tänka sig att det sker då (se figur nedan) eftersom
    att 4 transformerar hela grafen upp med 4. Men för att göra detta 
    rigoröst lösar man ut x
    $$y-4=\frac{2}{x-3}\Rightarrow x-3=\frac{2}{y-4}\Rightarrow x=\frac{2}{y-4}+3 $$

    Detta visar att y=4 är den andra asymptoten.
    
    \begin{center}
        \includegraphics[scale=0.5]{Figur5.png}
    \end{center}

    \item 
    I uppgiften får vi reda på två sanningar. Med följande variabler

    \begin{center}
        \includegraphics[scale=0.4]{Figur 6.png}
    \end{center}
    
    Får vi ut att Arean $A=ab$ och omkretsen $O=50=2a+b$
    Omkretsekvationen kan skrivas om som $b=50-2a$ och sedan 
    sättas in i den första vilket bildar $A=a(50-2a)=50a-2a^2$.

    Maxpunkten blir då där derivatan av funktionen är noll.
    $50-4a=0\Rightarrow a=12.5$ meter.

    Så när a är 12.5 meter blir det optimal area, och följande 
    ekvationen kring omkretsen måste b ha längden $50-2\cdot 12.5=25$ meter.

    \item 
    Med produktregeln blir derivatan
    $(x^2)'2^x+x^2(2^x)'=2x\cdot 2^x + x^2\cdot 2^x\cdot ln(2)$

    Och dom blir noll när 
    $$0=2x\cdot 2^x + x^2\cdot 2^x\cdot ln(2)$$
    $$-x^2\cdot 2^x\cdot ln(2)=2x\cdot 2^x $$
    $$-x^2\cdot ln(2)=2x$$
    $$e^{-x^2\cdot ln(2)}=e^{2x}$$
    $$e^{-x\cdot ln(2)}=e^{2}$$


\end{enumerate}
\end{document}