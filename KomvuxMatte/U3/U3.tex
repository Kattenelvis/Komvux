\documentclass[a4paper,12pt]{article}
%For images
\usepackage{graphicx}
 
\addtolength{\oddsidemargin}{-.875in}
\addtolength{\evensidemargin}{-.875in}
\addtolength{\textwidth}{1.75in}
 
\addtolength{\topmargin}{-.875in}
\addtolength{\textheight}{1.75in}
 
\begin{document}
\begin{enumerate}
    \item I uppgiften görs ett antagande att den även begränsas
    av y-axeln.

    \begin{center}
        \includegraphics[scale=0.4]{Figur 1.png}
    \end{center}

    Eftersom rotationen sker kring x-axeln så blir
    värdet på y radien, och då används formeln
    $$\pi\int_a^bf(x)^2dx$$
    I figuren så kan man observera att integralen går från 0 till 5
    $$\pi\int_0^5(x^2)^2dx=\pi\int_0^5x^4dx$$
    Som då integreras
    $$\Rightarrow \pi(\frac{x^5}{5})]^5_0=\pi 5^5/5=\pi 625 v.e$$

    \item Radien i denna uppgift blir istället radien x. Då hamnar x på $x=y^2$.
    Vilket grafiskt kan tänkas som att funktionerna $x^2$ och $\sqrt{x}$ har en rotationel
    symetri. Då används formeln 
    $$\pi\int_a^bf(y)^2dy$$
    \begin{center}
        \includegraphics[scale=0.4]{Figur 2.png}
    \end{center}
    
    I figuren ser vi att integralen går från 0 till 4.
    $$\pi\int_0^4(y^2)^2dy=\pi\int_0^4y^4dy$$
    $$\Rightarrow \pi\frac{y^5}{5}]^4_0=\pi \frac{4^5}{5}=\pi \frac{1024}{5} v.e$$


    \item Följande area ska utberäknas
    \begin{center}
        \includegraphics[scale=0.4]{Figur 3.png}
    \end{center}

    Med sådana integraler räknar man rektanglarna mellan funktionerna, dvs
    $g(x)-f(x)$, och integrerar sedan över den nya funktionen som i detta
    fall blir $$4cos(3x)-2sin(3x)$$

    Som då blir den här arean
    \begin{center}
        \includegraphics[scale=0.4]{Figur 4.png}
    \end{center}

    Integralen som räknas ut blir då mellan 0 och intersektionspunkten.
    Intersektionspunkten är när 
    $$4cos(3x)=2sin(3x)\Rightarrow 2=\frac{sin(3x)}{cos(3x)}=tan(3x)\Rightarrow x=\frac{tan^{-1}(2)}{3}$$
    
    $$\int_0^{tan^{-1}(2)/3}4cos(3x)-2sin(3x)dx=\frac{4sin(3x)+2cos(3x)}{3}]^{tan^{-1}(2)/3}_0$$
    $$=\frac{4sin(tan^{-1}(2))+2cos(tan^{-1}(2))-2}{3}\approx \frac{4sin(1.11)+2cos(1.11)-2}{3}\approx 0.824 a.e$$

    Detta är rimligt med tanke på att området är väldigt tunnt.

    \item Differentialekvationen kan skrivas om som $y'=-3y$. Som då kan integreras och lösas ut som
    $${dx}=\frac{1}{-3y}dy\Rightarrow x=-\frac{1}{3}ln(y)+C$$
\end{enumerate}
\end{document}