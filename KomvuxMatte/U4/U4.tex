\documentclass[a4paper,12pt]{article}
%For images
\usepackage{graphicx}
 
\addtolength{\oddsidemargin}{-.875in}
\addtolength{\evensidemargin}{-.875in}
\addtolength{\textwidth}{1.75in}
 
\addtolength{\topmargin}{-.875in}
\addtolength{\textheight}{1.75in}
 
\begin{document}
\begin{enumerate}
    \item \begin{enumerate}
              \item Komplexa tal läggs ihopp som vektorer.
                    $$(3+4i)+(5-i)=8+3i$$
              \item För $z_1$ så är radien $\sqrt{3^2+4^2}=5$ och för
                    $z_2$ är $\sqrt{5^2+1^2}=\sqrt{26}$

                    Vinkeln räknas ut med den inversa tangent funktionen
                    för lutningen av vektorn som pekar mot talet.
                    För $z_1$ är denna $\theta = tan^{-1}(\frac{4}{3})$
                    och för $z_2$ är denna $\theta = tan^{-1}(\frac{5}{-1})$
          \end{enumerate}
\end{enumerate}
\end{document}