\documentclass[a4paper,12pt]{article}
%For images
\usepackage{graphicx}
 
\addtolength{\oddsidemargin}{-.875in}
\addtolength{\evensidemargin}{-.875in}
\addtolength{\textwidth}{1.75in}
 
\addtolength{\topmargin}{-.875in}
\addtolength{\textheight}{1.75in}
 
\begin{document}
\begin{enumerate}
    \item \begin{enumerate}
              \item Komplexa tal läggs ihopp som vektorer.
                    $$(3+4i)+(5-i)=8+3i$$
              \item För $z_1$ så är radien $\sqrt{3^2+4^2}=5$ och för
                    $z_2$ är $\sqrt{5^2+1^2}=\sqrt{26}$

                    Vinkeln räknas ut med den inversa tangent funktionen
                    för lutningen av vektorn som pekar mot talet.
                    För $z_1$ är denna $\theta = tan^{-1}(\frac{4}{3})$
                    och för $z_2$ är denna $\theta = tan^{-1}(-\frac{1}{5})$

                  Så multiplicerar man dom

                  $$5e^{tan^{-1}(\frac{4}{3})}\sqrt{26}e^{tan^{-1}(-\frac{1}{5})}=5\sqrt{26}e^{tan^{-1}(4/3)+tan^{-1}(-1/5)}$$

                  \item från förra uppgiften vet vi att vineln för $z_2=tan^{-1}(-1/5)=arg(z_2)\approx -11.31$.
                  Denna vinkel är trovärdig om man kollar på bilden nere.
                  
                  \begin{center}
                        \includegraphics[scale=0.5]{Figur 1.png}
                  \end{center}

          \end{enumerate}
          \item \begin{enumerate}
                \item Med radien 5 och vinkeln $\pi/3$ så 
                skrivs den som $5e^{i\pi/3}$

                \item 
          \end{enumerate}

          \item denna kan skrivas om som 
          $z^2-6z+13=0$ vilket är en simpel andragradsekvation som
          kan lösas med pq-formeln 
          $$-\frac{-6}{2}\pm\sqrt{(\frac{-6}{2})^2-13}=3\pm\sqrt{-4}=3\pm 2i$$
          
          Grafiskt kan man se detta som två ställen där talen convergerar.
          Runt de områderna är dom nära noll, och presis vid $3\pm 2i$ är 
          transformeras den till noll punkten.
          \begin{center}
                \includegraphics[scale=0.6]{Figur 2.png}
          \end{center}
          \item 
          $$\frac{10(1-2i)}{1+2i(1-2i)}=\frac{10-20i}{1+4}=2-4i$$

          \item För att utföra detta skrivs formen om till a+bi form.
          
          
\end{enumerate}
\end{document}